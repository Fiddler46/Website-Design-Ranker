\documentclass[11pt]{beamer}
\usepackage[utf8]{inputenc}
\usepackage[T1]{fontenc}
\usepackage{lmodern}
\usetheme{Rochester}
\usepackage{graphicx}
\begin{document}
	\title{\textbf{Website Design Ranker}}
	\subtitle{Using Machine Learning}
	%\logo{}
	\institute{\large Department of Computer Science and Engineering \\\textbf{FISAT}}
	\date{16 SEPTEMBER 2019}
	\author{{\scriptsize Adhyaksh Guhan - 7 , Anet Eliza Johny - 23 , Dharwish Raj - 47 , \\ Joel J Padayattil - 60}}
	%\setbeamercovered{transparent}
	%\setbeamertemplate{navigation symbols}{}
	\begin{frame}[plain]
		\maketitle
	\end{frame}
	\begin{frame}{Introduction}
		\begin{itemize}
			\item Our project involves an Machine learning algorithm that uses certain parameters to rank websites in terms of their design.
			
			\item In this presentation, we will be explaining, comparing and contrasting existing works that are similar in purpose to our project.
		
			\item We will be comparing against:
				\item A Website that ranks other design according to submissions by critics
				\item A ranking implementation done by Google that analyses a webpage's contents
				\item An algorithm called CoLiDes models how people navigate a complex website to find information
		\end{itemize}
	\end{frame}
	\begin{frame}{Problem}
		\begin{itemize}
			\item Our main problem is to evaluate website designs using an algorithm that uses machine learning
			\item It must take into account various parts of the website to use as parameters.
		\end{itemize}
	\end{frame}
	\begin{frame}{Why "Website Design Ranker" ?}
			\begin{itemize}
			\item Website Design Ranker will rank set of input websites as choice of parameter.
			\item It will be helpful to find best website among list of websites which have same content.

		\end{itemize}
	\end{frame}
	\begin{frame}{Existing Methods : Website Design Ranking Agencies }
		\begin{itemize}
			\item Wesites are ranked by Ranking Agencies as per submission on their database.
			\item Hired critcs and anayalzing staffs are reviewd and ranked according to their policy.
			\item No automated, More time consuming\\\
			\\Example:\\
			https://www.awwwards.com\\

			https://www.cssdesignawards.com\\

			https://www.csswinner.com/winners\\

			https://thefwa.com\\
		\end{itemize}
	\end{frame}
	\begin{frame}
	\frametitle{{Existing Methods : Google Page Layout}}
	\begin{itemize}
		\item Google introduced Page Layout Alorithm to analyse website readability.[1]
		\item Looks for the layout of the wedpage and the amound of content we see in the page once we click on a result.[1]
		
		\item Focuses to reduce the difficulty of users to find the actual content.[1]
		
		\item  The websites which does not have a lot of visible content above-the-fold and dedicates a large fraction (above a normal degree) to ads will be affected.[1]
		
		
	\end{itemize}

	\end{frame}
\begin{frame}
	\frametitle{{Existing Methods : Google Page Layout}}
		\begin{figure}
		
		\includegraphics[width=10cm]{image/gpa.png}
		\caption{One of the criteria of GPL Algorithm\textsuperscript{[Fig:1]}}
		\label{fig1:gpa}
	\end{figure}
\end{frame}
\begin{frame}
	\frametitle{{Existing Methods : CoLiDes models}}
\end{frame}
	\begin{frame}{Problem Analysis}
		\end{frame}
		\begin{frame}{What we proposed?}
			\begin{itemize}
				\item We propose a system where an algorithm scrubs through a website, looking for various elements.
				\item Once we discover the nature of these elements, we check whether the parameters we have set (eg: colour, symmetry, etc) have been met.
				\item For each parameter met, a website will obtain a mark.
				\item Once all parameters have been checked, the website receieves an overall score (the sum of all marks) that ranks its design.
					\end{itemize}	
	\end{frame}
	\begin{frame}{Conclusion}
		\begin{itemize}
			\item Here we can see the logical differences in the approaches that our algorithm takes versus any existing methods.
		\end{itemize}
	\end{frame}
	\begin{frame}
		\frametitle{\LARGE \textbf{References}}
		\begin{itemize}
			\item [1] - Google Page Layout Algorithm: Everything You Need to Know 
			"https://www.searchenginejournal.com/google-algorithm-history/page-layout/close"
		\end{itemize}
	\end{frame}
\end{document}
