\documentclass[11pt]{beamer}
\usepackage[utf8]{inputenc}
\usepackage[T1]{fontenc}
\usepackage{lmodern}
\usetheme{Rochester}
\usepackage{graphicx}
\begin{document}
	\title{\textbf{Website Design Ranker}}
	\subtitle{Using Machine Learning}
	%\logo{}
	\institute{\large Department of Computer Science and Engineering \\\textbf{FISAT}}
	\date{14 NOVEMBER 2019}
	\author{{\scriptsize Adhyaksh Guhan - 7 , Anet Eliza Johny - 23 , Dharwish Raj - 47 , \\ Joel J Padayattil - 60}}
	%\setbeamercovered{transparent}
	%\setbeamertemplate{navigation symbols}{}
	\begin{frame}[plain]
		\maketitle
	\end{frame}
	\begin{frame}{Problem Statement}
		\begin{itemize}
			
			
			\item Our project is aimed at ranking websites in terms of its design which is evaluated baised on certain parameters.
			
			\item Since a perfect model for website ranking  is not in practice this  follows ranking according to submissions by critics.
			
		\end{itemize}
	\end{frame}
	\begin{frame}{Scope and Challenges}
		\begin{itemize}
			
			\item There was no existing methodology for mannualfor analyzing and evaluating websites.
			
			\item The method which followed until this time was baised on submissions by the critics.
			
			\item Our main problem in evaluate website designs was that each websites were of different layout sizes so it was hard to compare.
			
			\item Visibility, clarity and duplicate content also affected.
			
	
			
		\end{itemize}
	\end{frame}
	\begin{frame}{Proposed System}
			\begin{itemize}
			\item Website Design Ranker will rank set of input websites based on certain parameters.
			\item It will be helpful to find best website among list of websites which have same content.
			\item We can compare our website design with other competing websites.
			\item We can see how a website's design may improve in an area.

			\end{itemize}
	\end{frame}
	\begin{frame}{Explanation }
		\begin{itemize}
			\item 
		\end{itemize}
	\end{frame}
		\begin{frame}{Methodology used }
	\begin{itemize}
		\item 
	\end{itemize}
	\end{frame}
	\begin{frame}
	\frametitle{{Algorithm}}
	\begin{enumerate}
	\item Start
	\item Using a website scraper to accept the various website addresses
	\item Scraping through the source code of each website via CSS files
	\item Find the hex codes of all elements of the website and count them with a count variable
	\item If count == 0
	\item [(4.1)] Give mark as 0
	\item else if count > 5
	\item [(5.1)] Give mark as 0
	\item else if count <= 5
	\item [(6.1)] Give mark as 1
	\item Stop
	\end{enumerate}
	
	\end{frame}
	\begin{frame}
	\frametitle{{Current Status}}
		\begin{figure}
		
	\end{figure}
	\end{frame}
\begin{frame}
			\frametitle{{Project Completion Time}}
	\begin{itemize}
		\item
	\end{itemize}
\end{frame}

\begin{frame}{Experimental result}
			\begin{itemize}
				\item
			\end{itemize}	
	\end{frame}

	\begin{frame}{Social and ethical relevence}
		\begin{itemsize}
			\item
		\end{itemsize}
\end{frame}

	\begin{frame}{Conclusion}
		\begin{itemize}
			\item Here we can see the logical differences in the approaches that our algorithm takes versus any existing methods.
			\item Our method relies on an objective and automated method that is consistent in nature as opposed to the subjective methods of the existing methods.
		\end{itemize}
	\end{frame}
	\begin{frame}
		\frametitle{\LARGE \textbf{References}}
		\begin{itemize}
			\item [1] - Google Page Layout Algorithm: Everything You Need to Know 
			"https://www.searchenginejournal.com/google-algorithm-history/page-layout/close"
			\item [Fig:1] - https://cdn.searchenginejournal.com/wp-content/uploads/2017/10/google-algorithm-above-the-fold-380x238.png
		\end{itemize}
	\end{frame}
\end{document}
